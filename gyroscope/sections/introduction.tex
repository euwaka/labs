A gyroscope is a device that demonstrates the interplay between angular momentum, torque, precession, and nutation. The dynamics are governed by the principles of rotational motion, which are essential for explaining the behavior of rotating systems and their applications in various scientific and engineering fields. Therefore, the aim of this report is to determine whether the precession model yields the conservation of angular momentum in the gyroscope for different external torques and initial spinning angular frequencies. The hypothesis posits that the precession frequency will increase linearly with the applied torque. This relationship can be expressed as:

\begin{equation*}
    \Omega = frac{\tao}{L}
\end{equation*}

where  is the precession frequency,  is the applied torque, and L is the angular momentum of the gyroscope. It further considers the sources of errors, as well as potential improvements to ensure more precise numerical support for the hypothesis.
