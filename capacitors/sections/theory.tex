\section{Theory} \label{sec:theory}

The main subject of this lab report is the capacitor and the laws that govern it. Therefore, in this section, the principles of capacitors and conductors are described. More detailed text on capacitors and other essential topics of electromagnetism can be found in \cite{griffiths} as well as in \cite{manual} as seen in our references.

A \textit{conductor} is matter that contains free charges\footnote{In real life, those free charges are, for instance, electrons on the valence shell or ions detached from the lattice.}. Initially, the charges repel each other, so that they are all confined on the surface of the conductor. Therefore, no free charges are found in the volume of the conductor, meaning that the field inside is zero. Therefore, from $\mathbf{E} = -\nabla \phi$, the potential inside a conductor is constant. 

A capacitor is a collection of conductors charged to some charge. In this lab, the capacitor is made up of two parallel metal plates, one charged with some charge $Q$, which induces a charge $-Q$ on the negative plate. Since the two conductors are equipotential in volume, one can speak of an unambiguous potential difference (voltage) of the capacitor, $V_C$. The relation between the charge and the voltage is given by \cite{griffiths}:

\begin{equation*}
    Q = CV
\end{equation*}

The capacitance is an intrinsic property of a capacitor, and depends solely on its geometry. For a parallel-plate capacitor, the capacitance can be found to be: \cite{griffiths}

\begin{equation*}
    C = \frac{\epsilon_0 A}{d}
\end{equation*}

When a capacitor is charged to some voltage $V$, the energy stored in the capacitor is \cite{manual}

\begin{equation*}
    W = \frac{1}{2}CV^2
\end{equation*}
