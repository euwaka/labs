\section{Experimental data from the lab} \label{appendix:data}

\textbf{The capacitor as a function of charge}. The voltage of the capacitor ($V_C$) has been measured for the number $q$ of applied charges at the separations $s_1 = 2 (mm)$ and $s_2 = 4 (mm)$ of the capacitor plates. There are $10$ measurements\footnote{Thus, maximally $10$ applied charges to the capacitor plate.}, each sampled $4$ times. The data is displayed in Table \ref{tab:appendix:data:exp1}

\begin{table}[H]
    \centering
    \begin{minipage}{0.4\textwidth}
        \centering
        \begin{tabular}{|c|cccc|}
            \hline
            $q$ & $V_{C1}, (V)$ & $V_{C2}, (V)$ & $V_{C3}, (V)$ & $V_{C4}, (V)$ \\
            \hline
            1  & 0.5 & 0.8 & 0.8 & 0.5 \\
            2  & 1.5 & 2.0 & 1.6 & 1.8 \\
            3  & 3.0 & 2.5 & 2.2 & 2.5 \\
            4  & 4.0 & 3.5 & 2.5 & 3.0 \\
            5  & 4.5 & 4.0 & 3.4 & 3.5 \\
            6  & 5.9 & 4.5 & 3.8 & 4.5 \\
            7  & 6.6 & 5.0 & 4.0 & 5.0 \\
            8  & 7.4 & 5.5 & 4.5 & 5.5 \\
            9  & 8.4 & 5.8 & 5.0 & 5.8 \\
            10 & 9.0 & 6.2 & 5.5 & 5.5 \\
            \hline
        \end{tabular}
        \caption{$s_1=2 (mm)$ separation}
    \end{minipage}
    \hspace{2cm}
    \begin{minipage}{0.4\textwidth}
        \centering
        \begin{tabular}{|c|cccc|}
            \hline
            $q$ & $V_{C1}, (V)$ & $V_{C2}, (V)$ & $V_{C3}, (V)$ & $V_{C4}, (V)$ \\
            \hline
            1   & 3.0  & 3.0  & 4.0  & 3.0  \\
            2   & 6.0  & 5.8  & 4.5  & 7.0  \\
            3   & 8.5  & 8.5  & 7.0  & 8.5  \\
            4   & 12.0 & 12.0 & 9.0  & 8.0  \\
            5   & 13.0 & 15.5 & 12.0 & 10.0 \\
            6   & 15.0 & 18.0 & 15.0 & 12.0 \\
            7   & 17.0 & 20.0 & 16.5 & 15.0 \\
            8   & 18.0 & 22.5 & 18.0 & 12.0 \\
            9   & 19.0 & 25.0 & 20.0 & 15.0 \\
            10  & 20.0 & 27.0 & 24.0 & 16.0 \\
            \hline
        \end{tabular}
        \caption{$s_2=4 (mm)$ separation}
    \end{minipage}

    \caption{$V_C$ as the number $q$ of applied charges for separations $s_1 = 2 (mm)$ and $s_2 = 4(mm)$}
    \label{tab:appendix:data:exp1}
\end{table}

\textbf{The distribution of charge on a surface}. $11$ measurements were made by going radially from the center of the capacitor plate, starting at $0 (cm)$ to the very edge of the plate ($10 (cm)$). Each measurement was sampled $3$ times. The data is displayed in Table \ref{tab:appendix:data:exp2}.
\begin{table}[H]
    \centering
    \begin{tabular}{|c|ccc|}
        \hline
         $r, (m)$ & $V_1$ & $V_2$ & $V_3$ \\
         \hline
         0.00 & 2.9 & 3.3 & 3.3 \\
         0.01 & 3.0 & 3.2 & 3.5 \\
         0.02 & 2.5 & 3.4 & 3.0 \\
         0.03 & 3.2 & 3.1 & 3.5 \\
         0.04 & 2.8 & 3.3 & 3.4 \\
         0.05 & 3.0 & 3.5 & 3.0 \\
         0.06 & 3.5 & 3.6 & 4.0 \\
         0.07 & 4.0 & 4.0 & 4.3 \\
         0.08 & 4.6 & 4.5 & 4.5 \\
         0.09 & 6.0 & 5.5 & 6.0 \\
         0.10 & 6.5 & 6.6 & 6.4 \\
         \hline
    \end{tabular}
    \caption{Charge density data}
    \label{tab:appendix:data:exp2}
\end{table}

\textbf{The potential difference as function of the plate distance at constant charge}. $13$ measurements were made starting from the initial separation $s_1 = 2(mm)$ all the way up to the maximum possible separation in the setup - $s_{max} = 12 (cm)$. Each measurement was sampled $4$ times using the $100V$ range of the electrometer. The data is displayed in Table \ref{tab:appendix:data:exp3}.
\begin{table}[H]
    \centering
    \begin{tabular}{|c|ccccc|}
        \hline
        $s, (m)$ & $V_1, (V)$ & $V_2, (V)$ & $V_3, (V)$ & $V_4, (V)$ & $V_5, (V)$ \\
        \hline
        0.002 & 20.0 & 20.0 & 20.0 & 20.0 & 20.0 \\
        0.01  & 40.0 & 40.0 & 42.0 & 41.0 & 41.0 \\
        0.02  & 45.0 & 60.0 & 50.0 & 42.0 & 50.0 \\
        0.03  & 50.0 & 55.0 & 55.0 & 45.0 & 50.0 \\
        0.04  & 45.0 & 58.0 & 57.0 & 55.0 & 60.0 \\
        0.05  & 58.0 & 60.0 & 57.0 & 49.0 & 60.0 \\
        0.06  & 59.0 & 60.0 & 45.0 & 45.0 & 50.0 \\
        0.07  & 55.0 & 55.0 & 55.0 & 61.0 & 55.0 \\
        0.08  & 55.0 & 55.0 & 56.0 & 55.0 & 50.0 \\
        0.09  & 60.0 & 62.0 & 59.0 & 60.0 & 60.0 \\
        0.10  & 62.0 & 63.0 & 55.0 & 45.0 & 50.0 \\
        0.11  & 63.0 & 65.0 & 58.0 & 60.0 & 50.0 \\
        0.12  & 65.0 & 70.0 & 50.0 & 56.0 & 50.0 \\
        \hline
    \end{tabular}
    \caption{Voltage of the $20V$-charged capacitor at different separations $s \in \left[ s_1, s_{max}\right]$}
    \label{tab:appendix:data:exp3}
\end{table}