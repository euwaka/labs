\documentclass[a4paper,12pt]{article}

\usepackage{float}
\usepackage{amsmath}
\usepackage{amsfonts}
\usepackage{amssymb}
\usepackage{graphicx}
\usepackage[margin=0.8in]{geometry}
\usepackage[utf8]{inputenc}

% Title information
\title{Oscillations}
\author{Artur Topal (S5942128), Tyn Rendering (S6106366)}
\date{11 December 2024}

\begin{document}

\maketitle

\begin{abstract}
This experiment aswers the question: What is the relation between the damping factor and the resonance frequency in a damped driven oscillator system? The natural frequency of an undamped oscillator is measured, damping is introduced by a magnet and a driving force is applied. The amplitude and phase are recorded at different driving frequencies and used to construct amplitude-frequency and phase-frequency graphs. The relation that we found between the damping factor and resonance frequency was that by an increase in the damping factor lead to the maximum amplitude being reduced and the resonance frequency to shift slightly below the natural frequency.
\end{abstract}

\section{Introduction}
This is our \textbf{very short} introduction.

\section{Theory}

This is \textbf{brief statement of experimental goal and only give the formulae relevant for analysis (do not give long theoretical introductions: for these reports it is allowed and encouraged to refer to the manual.}


\section{Experimental setup}
The main objective of this experiment was to investigate the oscillatory motion of a damped harmonic oscillator that was driven by an electric motor. The experimental setup depicted in Figure 1 consists of an oscillation disc at the top, that is equipped with a rotation sensor to measure the angular displacement. A magnet that adds a damping term is located next to the disc. An electric motor at the bottom of the setup delivers a driving force. A second rotary motion sensor is located above the motor to measure the frequency and phase of the driving force.

\begin{figure}[h!]
    \centering
    \includegraphics[width=0.8\textwidth]{oscillations/images/setup}
    \caption{The experimental setup}
    \label{fig:setup}
\end{figure}

During the first phase of the experiment, the goal was to determine the natural frequency of the oscillator. To accomplish this, the magnet was first removed and there was no driving force being supplied. Instead, the disc was manually rotated and its oscillations were recorded.
After these measurements the magnet was reinstalled to introduce a damping term, the magnet's position was adjusted so the oscillations would be damped after about 10 periods.

In the second phase, a driving force was supplied by the electrical motor. First, a couple of measurements of the amplitude of the oscillations and the frequency of the driving force were taken to gauge the voltage range that is to be used in finding the peak amplitude. After this, two measurements were taken at seventeen different voltage levels. Most of these measurements were conducted around the resonance frequency to get a more precise determination of the peak amplitude.



\section{Results}
\subsection{Direct calculation of the damping factor.}

The oscillations produced without added damping factor (i.e. the magnet) are displayed in Figure \ref{fig:undamped_oscillations}. The assumption was made that the string to which the springs were attached did not slip from the pulley.

\begin{figure}[h!]
  \centering
  \includegraphics[width=0.6\textwidth]{oscillations/images/Undamped_Oscillations}
  \caption{Oscillations without damping}
  \label{fig:undamped_oscillations}
\end{figure}

From the measurements, the coordinates of the maxima were determined and from this the natural frequency was determined to be $\omega = 4.8 \pm 0.1 (rad/s)$, which was constant throughout the measurements.

\begin{figure}[h!]
  \centering
  \includegraphics[width=0.6\textwidth]{oscillations/images/underdamped}
  \caption{Decrease in amplitude by an added damping term}
  \label{fig:underdamped}
\end{figure}

A damping term was added by placing the magnet back in its original position, the oscillations are displayed in Figure \ref{fig:underdamped}. The coordinates of nine of the maxima were determined and were plotted in Figure \ref{fig:log of damping factor}. 

\begin{figure}[h!]
  \centering
  \includegraphics[width=1\textwidth]{oscillations/images/unnamed}
  \caption{Semi-logaritmic plot of the amplitude over time}
  \label{fig:log of damping factor}
\end{figure}

The displacement amplitude $x(t)$ decreases as shown in Eq.~\eqref{eq:underdamped_solution}. The slope of the line of best-fit $m$ is equal to the $-\gamma$ term in Eq.~\eqref{eq:underdamped_solution} ($|m| = \gamma $). By this a damping factor of $\gamma_{direct} = 1.3 \pm 0.2 (rad/s)$ was determined.

\subsection{Indirect calculation of the damping factor.}

\begin{figure}[H]
  \centering
  \includegraphics[width=1\textwidth]{oscillations/images/resonance}
  \caption{The amplitude and phase difference of the oscillations plotted against driving frequency.}
  \label{fig:resonance}
\end{figure}

The graph of the amplitude and phase difference of the oscillations against driving frequencies with best-fit curves and correspoding uncertanties are depicted in Figure.\ref{fig:resonance}. Tabular data with respective errors are depicted in Table \ref{tab:data} in appendix \ref{appendix:data}.

Resonance frequency from the amplitude-frequency graph is estimated to be $\omega_{amplitude} = 4.35 \pm 0.09 (rad/s)$. Resonance frequency from the phase-frequency graph is estimated to be $\omega_{phase} = 4.46 \pm 0.02 (rad/s)$. Combined resonance frequency is

\begin{equation*}
\omega_{res} = \frac{\omega_{amplitude} + \omega_{phase}}{2} \pm \sqrt{\Delta\omega_{amplitude}^2 + \Delta\omega_{phase}^2} = 4.4 \pm 0.1 (rad/s)
\end{equation*}

Damping factor is again evaluated from Eq.~\ref{eq:freq_deps}.

\begin{equation*}
        \gamma_{indirect} = \sqrt{ \frac{\omega^2 - \omega_{res}^2}{2} } = \sqrt{ \frac{4.8^2 - 4.4^2}{2} } = 1.3 \pm 0.8 (rad/s)
\end{equation*}       

This damping factor matches the damping factor previously calculated, $\gamma_{direct}$.

Error analysis for the resonance frequencies and the damping factor is located in Appendix \ref{appendix:errors}.


\section{Discussion}
Our discussion of the results of the experiment.


\section{Conclusions}
The outcome of this experiment was that adding a damping factor to the oscillating system decreased the amplitude of the resonance frequency and shifted its peak to the left.

Future improvements of this experiment would be to perform the experiment of a more stable base such that no superfluous motion would disrupt the oscillatory motion. Another improvement would be to use rotary sensors with smaller interfalls between measurements. Slipping of the strings should be minimized by restricting the angular moting of the springs and strings.


\appendix
\section{Preparatory Exercises} \label{appendix:preps}
\subsection{Question 1}

A change in $\gamma$ will cause a change in amplitude and a shift in resonance frequency. When $\gamma$ increases, the amplitude will decrease, because the amplitude is inversely proportional to $\gamma$ (see Eq.~\eqref{eq:full_motion_ampl}). The peak will be shifted to the left, since an increase in $\gamma$ causes a decrease in $\omega_{\text{res}}$, as seen in Eq.~\eqref{eq:freq_deps}. A decrease in $\gamma$ will have an inverse effect, the amplitude will increase and the peak will shift to the right.

\subsection{Question 2}
The amplitude was given by Eq.~\eqref{eq:full_motion_ampl}. The resonance frequency occurs when the amplitude A is maximized, this occurs when the denominator in Eq.~\eqref{eq:full_motion_ampl} is minimized. For this we take the derivative of the denominator with respect to $\omega_d$ and set it to zero. The resonance frequency is then determined to be: 
\begin{equation} \label{eq:freq_deps}
        \omega_{\text{res}} = \sqrt{\omega^2 - 2\gamma^2}
\end{equation}
From this follows that $\omega_{\text{res}} = \omega$ when $\gamma = 0$, this occurs when no damping is taking place.

\subsection{Question 3}
Note, that $\phi$ in Eq.~\eqref{eq:full_motion_phase} can be re-written as follows:
\begin{equation*}
        \phi = \arctan(\frac{2\gamma\omega_d}{\omega^2 - \omega_d^2}) = \arctan(\gamma ( \frac{1}{\omega-\omega_d} - \frac{1}{\omega+\omega_d} ))
\end{equation*}
Then, taking both half-limits of $\phi$ as $\omega_d \rightarrow \omega$ gives discontinuity \footnote{$arctan(\theta)$ is continuous on $(-\frac{\pi}{2};\frac{\pi}{2})$, so the limit operator can be brought inside $arctan(\theta)$.}:
\begin{equation*}
        \arctan \lim_{\omega_d \rightarrow \omega^+} {\gamma ( \frac{1}{\omega-\omega_d} - \frac{1}{\omega+\omega_d} )} \asymp \arctan( -\infty) = -\frac{\pi}{2}
\end{equation*}

\begin{equation*}
        \arctan \lim_{\omega_d \rightarrow \omega^-} {\gamma ( \frac{1}{\omega-\omega_d} - \frac{1}{\omega+\omega_d} )} \asymp \arctan(\infty) = \frac{\pi}{2}
\end{equation*}

\begin{figure}[H]
  \centering
  \includegraphics[width=1\textwidth]{oscillations/images/prep_exercise_Q3}
  \caption{Plots of the phase $\phi$ for different values of the damping factor $\gamma$ and natural frequency $\omega$.} 
  \label{fig:prep:phase}
\end{figure}

In Fig.~\ref{fig:prep:phase}, this discontinuity is clear for different initial conditions ($\gamma$ and $\omega$). If the driving frequency $\omega_d$ approaches natural frequency $\omega$ from the negative side, the phase $\phi$ approaches $\pi/2$ radians. In other words, the driving force becomes more out of phase with the natural oscillations. Identical reasoning applies to the positive half limit.

 
\section{Tabular Data} \label{appendix:data}
Two sets of measurements were taken for each experiment. The resulting data was calculated by taking the mean, and the error by taking the half absolute difference. 

\begin{table}[H]
\centering
\caption{Experimental Data}
\label{tab:data}
\begin{tabular}{cccc}
\hline
\textbf{Input Voltage (V)} & \textbf{Amplitude (rad)} & \textbf{Driving Frequency (rad/s)} & \textbf{Phase (rad)} \\
\hline
3.00 & $0.68 \pm 0.03$ & $2.54 \pm 0.2$   & $0.38 \pm 0.03$ \\
3.30 & $0.84 \pm 0.01$ & $3.06 \pm 0.02$  & $0.31 \pm 0.002$ \\
3.50 & $0.96 \pm 0.08$ & $3.30 \pm 0.06$  & $0.41 \pm 0.08$ \\
3.79 & $1.36 \pm 0.02$ & $3.66 \pm 0.01$  & $0.37 \pm 0.001$ \\
4.09 & $2.78 \pm 0.3$  & $4.03 \pm 0.03$  & $0.40 \pm 0.003$ \\
4.34 & $5.90 \pm 0.2$  & $4.34 \pm 0.01$  & $0.98 \pm 0.1$ \\
4.04 & $2.14 \pm 0.2$  & $3.94 \pm 0.07$  & $0.40 \pm 0.007$ \\
3.95 & $1.74 \pm 0.06$ & $3.82 \pm 0.02$  & $0.29 \pm 0.1$ \\
4.00 & $2.14 \pm 0.08$ & $3.93 \pm 0.004$ & $0.39 \pm 0.0004$ \\
5.00 & $2.67 \pm 0.1$  & $4.93 \pm 0.04$  & $-0.42 \pm 0.02$ \\
4.40 & $5.22 \pm 0.07$ & $4.34 \pm 0.09$  & $1.18 \pm 0.3$ \\
4.32 & $4.99 \pm 0.8$  & $4.25 \pm 0.06$  & $0.85 \pm 0.2$ \\
4.39 & $5.96 \pm 0.002$& $4.39 \pm 0.01$  & $1.21 \pm 0.1$ \\
4.41 & $5.44 \pm 0.5$  & $4.34 \pm 0.1$   & $1.14 \pm 0.4$ \\
5.20 & $1.66 \pm 0.06$ & $5.26 \pm 0.02$  & $-0.24 \pm 0.01$ \\
5.40 & $1.37 \pm 0.1$  & $5.41 \pm 0.06$  & $-0.29 \pm 0.1$ \\
4.44 & $5.87 \pm 0.07$ & $4.25 \pm 0.03$  & $0.75 \pm 0.1$ \\
\end{tabular}
\end{table}


\section{Errors} \label{appendix:errors}
\subsection{ Resonance Frequency from the Amplitude Graph } \label{appendix:errors:res_ampl}

In this case, resonance frequency was estimated by finding the peak value of the best-fit curve $A(x, a, b, c) = \frac{a}{\sqrt{(b-x^2)^2 + cx^2}}$. This is done by setting $\frac{\partial A}{\partial x} = 0$. Therefore,

\begin{equation*}
  \frac{\partial}{\partial x} A(x, a, b, c) = \frac{ax(2b-c-2x^2)}{( b^2 + x^2 (-2b + c + x^2) )^{3/2}} = 0 \Rightarrow x = \sqrt{b - c/2}
\end{equation*}

Values for $a$, $b$, and $c$ were obtained using standard statistical optimization techniques that also yielded the covarience matrix for $a$, $b$, and $c$. The values of interest are:
\begin{equation*}
  a = 8.40, b = 19.02, c = 0.10, \sigma_b = 0.18, \sigma_c = 0.02
\end{equation*}

Therefore, $x = \omega_{amplitude} = \sqrt{19.02 - 0.10 /2} \approx 4.35 (rad/s)$. Error is propagated using:
\begin{equation*}
  \frac{\sigma_{x}}{x} = \sqrt{ \left( \frac{\partial x}{\partial b} \sigma_b \right)^2 + \left( \frac{\partial x}{\partial c} \sigma_c \right)^2} = \sqrt{ \left( \frac{1}{2\sqrt{b - c/2}} \sigma_b \right)^2 + \left( \frac{1}{-4\sqrt{b-c/2}} \sigma_c \right)^2} 
\end{equation*}

Plugging the values in, error in the resonance frequency can be computed to be $\sigma_{x} = 0.09 (rad/s)$.

Therefore, $\omega_{amplitude} = 4.41 \pm 0.09 (rad/s)$.

\subsection{ Resonance Frequency from the Phase Graph } \label{appendix:errors:res_phase}

In this case, resonance frequency was estimated by finding the point of discontinuity of the best-fit curve $\phi(x, a, b) = \arctan \frac{ax}{b - x^2}$. For this function, the point of discontinuity coincides with the peak value, so the same statistical optimization techniques were utilized as in section \ref{appendix:errors:res_ampl}. The values and errors of the parameters are:
\begin{equation*}
  a=0.48, b = 19.93, \sigma_b = 0.18
\end{equation*}

Resonance frequency is located where $\phi = \pi/2$, thus in the peak-discontinuity point. This equality holds if $b-x^2=0$. Therefore, $x = \sqrt{b} > 0, x = 4.46 (rad/s)$. The error is computed as follows:

\begin{equation*}
  \frac{\sigma_x}{x} = \frac{\partial x}{\partial b} \sigma_b \Rightarrow \sigma_x = \frac{\sigma_b}{2\sqrt{b}} = 0.02
\end{equation*}

Therefore, $\omega_{phase} = 4.46 \pm 0.02 (rad/s)$.

\subsection{Damping Factor in terms of Resonance Frequency}

The relation between the damping factor $\gamma$ and natural frequency $\omega$ and the resonance frequency $\omega_{res}$ is
\begin{equation*}
  \gamma = \sqrt{\frac{\omega^2 - \omega_{res}^2}{2}} = 1.32 (rad/s)
\end{equation*}

The error is
\begin{equation*}
  \left( \frac{\sigma_\gamma}{\gamma} \right)^2 = \left( \frac{\partial \gamma}{\partial \omega} \sigma_\omega \right)^2 + \left( \frac{\partial \gamma}{\partial \omega_{res}} \sigma_{\omega_{res}} \right)^2 = \frac{\omega^2 \sigma_{\omega}}{2(\omega^2 - \omega_{res}^2)} + \frac{\omega_{res}^2\sigma_{\omega_{res}}}{2(\omega^2 - \omega_{res}^2)}
\end{equation*}

\begin{equation*}
  \sigma_\gamma = \gamma \sqrt{\frac{\omega^2 \sigma_{\omega}}{2(\omega^2 - \omega_{res}^2)} + \frac{\omega_{res}^2\sigma_{\omega_{res}}}{2(\omega^2 - \omega_{res}^2)}} = \gamma \sqrt{\frac{\omega^2 \sigma_\omega + \omega_{res}^2 \sigma_{\omega_{res}}}{2(\omega^2 - \omega_{res}^2)}} = 1.32 \sqrt{ \frac{4.78^2 \times 0.02 + 4.41^2 \times 0.09}{2(4.78^2 - 4.41^2)} } = 0.8
\end{equation*}

Therefore,
\begin{equation*}
  \gamma = 1.32 \pm 0.8 (rad/s) 
\end{equation*}


\end{document}
