A damped driven oscillatory system has a peak amplitude and the phase abruptly breaks at a specific resonance frequency. This resonance frequency depends on intrinsic constant natural angular frequency of the oscillations, and the damping factor. Because driven damped oscillatory systems are prevalent in modern physical and chemical models and characteristic resonance frequency successfully explains many physical and chemical phenomena, relation between the resonance frequency and the damping factor is essential in research. Therefore, this report aims to find the relation between the damping factor and the resonance frequency in such a system. The hypothesis of this report is the mathematical relation $\omega_{res} = \sqrt{\omega^2 - 2\gamma^2}$ which is used to compare damping factors of the same system found in two distinct ways, thus providing emperical grounds to the hypothesis. It also discusses the sources of errors, and relevant improvements needed to make more accurate numerical support for the hypothesis.
