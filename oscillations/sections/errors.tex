\subsection{Natural Frequency from the Amplitudes} \label{appendix:errors:fn_amplitude}

The uncertanty in $T_i$ is given by:
\begin{equation}
  \Delta T = (\frac{timing uncertainty\sqrt(N )})
\end{equation}

The error in the natural frequency gets propagated as:

\begin{equation}
  \Delta f_n = f_n^2\Delta T 
\end{equation}

\subsection{The damping factor from an underdamped oscillator}
In this case, $\Delta \gamma \propto \Delta t_i +\Delta m + \Delta A_i$. Here $\Delta t_i$ is the assigned error of $0.05$s, this being the measurement interfalls. The line of best fit is in the form $y=mx+c$, the uncertainty in the slope is given by:

\begin{equation}
  \Delta m = \sqrt{\frac{\sigma_y^2}{\sum (t_i - \bar{t_i})^2}}
\end{equation}
Where $\sigma_y^2$ denotes the variance in $A_i$, with $\sigma_y^2$ being equal to:

\begin{equation}
\sigma = \sqrt{\sum_{i=1}^{N} {(t_i - \bar{t_i})^2}{f(t_i)}}
\end{equation}

The uncertainty in $A_i$ is calculated by:
\begin{equation}
  s = \sqrt{\frac{1}{N-1}\sum_{i = 1}^{N}(t_i - \bar{t_i})^2}
\end{equation}

Combing the previous stated uncertanties lead to a value of $\Delta \gamma$ that is determined by:
\begin{equation}
  \Delta \gamma = \sqrt{(\Delta m)^2 + \sum_{i=1}^{N} \left[ \left( \frac{\Delta A_i}{t_i} \right)^2 + \left( \gamma \cdot \frac{\Delta t_i}{t_i} \right)^2 \right]}
\end{equation}

\subsection{ Resonance Frequency from the Amplitude Graph } \label{appendix:errors:res_ampl}

In this case, resonance frequency was estimated by finding the peak value of the best-fit curve $A(x, a, b, c) = \frac{a}{\sqrt{(b-x^2)^2 + cx^2}}$. This is done by setting $\frac{\partial A}{\partial x} = 0$. Therefore,

\begin{equation*}
  \frac{\partial}{\partial x} A(x, a, b, c) = \frac{ax(2b-c-2x^2)}{( b^2 + x^2 (-2b + c + x^2) )^{3/2}} = 0 \Rightarrow x = \sqrt{b - c/2}
\end{equation*}

Values for $a$, $b$, and $c$ were obtained using standard statistical optimization techniques that also yielded the covarience matrix for $a$, $b$, and $c$. The values of interest are:
\begin{equation*}
  a = 8.40, b = 19.02, c = 0.10, \sigma_b = 0.18, \sigma_c = 0.02
\end{equation*}

Therefore, $x = \omega_{amplitude} = \sqrt{19.02 - 0.10 /2} \approx 4.35 (rad/s)$. Error is propagated using:
\begin{equation*}
  \frac{\sigma_{x}}{x} = \sqrt{ \left( \frac{\partial x}{\partial b} \sigma_b \right)^2 + \left( \frac{\partial x}{\partial c} \sigma_c \right)^2} = \sqrt{ \left( \frac{1}{2\sqrt{b - c/2}} \sigma_b \right)^2 + \left( \frac{1}{-4\sqrt{b-c/2}} \sigma_c \right)^2} 
\end{equation*}

Plugging the values in, error in the resonance frequency can be computed to be $\sigma_{x} = 0.09 (rad/s)$.

Therefore, $\omega_{amplitude} = 4.41 \pm 0.09 (rad/s)$.

\subsection{ Resonance Frequency from the Phase Graph } \label{appendix:errors:res_phase}

In this case, resonance frequency was estimated by finding the point of discontinuity of the best-fit curve $\phi(x, a, b) = \arctan \frac{ax}{b - x^2}$. For this function, the point of discontinuity coincides with the peak value, so the same statistical optimization techniques were utilized as in section \ref{appendix:errors:res_ampl}. The values and errors of the parameters are:
\begin{equation*}
  a=0.48, b = 19.93, \sigma_b = 0.18
\end{equation*}

Resonance frequency is located where $\phi = \pi/2$, thus in the peak-discontinuity point. This equality holds if $b-x^2=0$. Therefore, $x = \sqrt{b} > 0, x = 4.46 (rad/s)$. The error is computed as follows:

\begin{equation*}
  \frac{\sigma_x}{x} = \frac{\partial x}{\partial b} \sigma_b \Rightarrow \sigma_x = \frac{\sigma_b}{2\sqrt{b}} = 0.02
\end{equation*}

Therefore, $\omega_{phase} = 4.46 \pm 0.02 (rad/s)$.

\subsection{Damping Factor in terms of Resonance Frequency}

The relation between the damping factor $\gamma$ and natural frequency $\omega$ and the resonance frequency $\omega_{res}$ is
\begin{equation*}
  \gamma = \sqrt{\frac{\omega^2 - \omega_{res}^2}{2}} = 1.32 (rad/s)
\end{equation*}

The error is
\begin{equation*}
  \left( \frac{\sigma_\gamma}{\gamma} \right)^2 = \left( \frac{\partial \gamma}{\partial \omega} \sigma_\omega \right)^2 + \left( \frac{\partial \gamma}{\partial \omega_{res}} \sigma_{\omega_{res}} \right)^2 = \frac{\omega^2 \sigma_{\omega}}{2(\omega^2 - \omega_{res}^2)} + \frac{\omega_{res}^2\sigma_{\omega_{res}}}{2(\omega^2 - \omega_{res}^2)}
\end{equation*}

\begin{equation*}
  \sigma_\gamma = \gamma \sqrt{\frac{\omega^2 \sigma_{\omega}}{2(\omega^2 - \omega_{res}^2)} + \frac{\omega_{res}^2\sigma_{\omega_{res}}}{2(\omega^2 - \omega_{res}^2)}} = \gamma \sqrt{\frac{\omega^2 \sigma_\omega + \omega_{res}^2 \sigma_{\omega_{res}}}{2(\omega^2 - \omega_{res}^2)}} = 1.32 \sqrt{ \frac{4.78^2 \times 0.02 + 4.41^2 \times 0.09}{2(4.78^2 - 4.41^2)} } = 0.8
\end{equation*}

Therefore,
\begin{equation*}
  \gamma = 1.32 \pm 0.8 (rad/s) 
\end{equation*}
